\section{Install}
You can check out the code by doing
\begin{align*}
	\texttt{\$ svn checkout https://ssftrac.ugent.be/svn/wim .}
\end{align*}
When first building the code make sure you already have root installed with the modules \texttt{fftw3}, \texttt{gviz}, \texttt{mathmore} and \texttt{minuit2} enabled. To build and install the code do
\begin{align*}
    & \texttt{\$ cd build} \\
	& \texttt{\$ cmake -DCMAKE\_BUILD\_TYPE=RELEASE ..} \\
	& \texttt{\$ make } \\
	& \texttt{\$ make install } 
\end{align*}
\subsection{Troubleshooting}
\begin{itemize}
\item Error in \texttt{cmake}: ROOTSYS not set.
If you get an error like this you probably should add the path of the \texttt{root-config} script to the search path in the file \texttt{\$\{PROJECT\_SOURCE\_DIR\}/cmake/Modules/FindROOT.cmake}. This is done by adding the path to the line \texttt{SET(ROOT\_CONFIG\_SEARCHPATH /insert\_your\_path\_here)}. In my case this was \texttt{usr/local/bin}. If you don't know the location of the \texttt{root-config} executable you maybe able to find it using \texttt{\$ find /usr -name root-config}.
\item Error \texttt{Wrong ELF class}. If you get errors mentioning ELF-class then you are probably linking 64 to 32 bit code which does not compute. Make sure you are linking to the 64 bit libraries of \texttt{root} and not the 32 bit ones. For me the 64 bit libraries were located in \texttt{/usr/local/lib/root} while the 32 bit ones were in \texttt{/usr/local/root/lib/root}. 
\end{itemize}