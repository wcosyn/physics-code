\section{Install}
You can check out the code by doing (with \texttt{<dir>} the directory where 
you want the code to end up)
\begin{align*}
	\texttt{\$ svn checkout https://ssftrac.ugent.be/svn/wim <dir>}
\end{align*}
When first building the code make sure you already have \texttt{root} installed 
with the modules \texttt{mathmore} and 
\texttt{minuit2} enabled.  Also the \texttt{gsl} libraries need to be 
installed. \texttt{Cmake} should give an error if this is not the case when 
trying to build. During configuration, \texttt{cmake} currently doesn't check 
the required extra \texttt{root} libraries I think.  You can do this yourself 
by executing \texttt{\$ root-config --has-<feature>} (where you replace 
\texttt{<feature>} with \texttt{mathmore} etc). To build the documentation you 
also need \texttt{doxygen} and \texttt{graphviz} packages installed. To build 
and install the code do (in the dir where you checked 
out the code)
\begin{align*}
    & \texttt{\$ cd build} \\
	& \texttt{\$ cmake -DCMAKE\_CXX\_COMPILER=g++}\\
 &\qquad \texttt{-DCMAKE\_C\_COMPILER=gcc 
-DCMAKE\_BUILD\_TYPE=Release ..} \\
	& \texttt{\$ make -jN}\quad [\texttt{N}= \text{number of cores you want 
to use}]\\
	& \texttt{\$ make install } \\
	& \texttt{\$ make doc } 
\end{align*}
If you have the intel C and C++ compilers installed, substitute \texttt{g++} 
with \texttt{icpc} and \texttt{gcc} with \texttt{icc}.  Other build types than 
\texttt{Release} include \texttt{Debug, Profile} (there's more, see the 
\texttt{CMakeLists.txt} file in the main dir for those if you want to use 
them). 
Note that currently \texttt{CMake} version 2.8.10 is required.  If you want to 
have static libraries built (the default is only dynamic ones), you can add 
\texttt{-DSTATIC\_BUILD=yes} to the first commandline.
% This is not yet a 
% standard module on the UGent HPC so don't forget to do \texttt{\$> module load 
% CMake/2.8.10} when you want to build the Code.  
If your hostname is not one of 
the predefined ones (Wim's pc \& laptop, the HPC of UGent) you will end up 
using the default compiler flags, which should be fine in most cases.  In case 
you want to add your custom compiler flags, you can add an extra case in the 
\texttt{CMakeLists.txt} file in the root dir of the code.  The last command 
\texttt{make doc} generates the API through DoxyGen.  These can be accessed in 
the \texttt{doc/html/index.html} file [html] or by compiling 
\texttt{doc/latex/refman.text} [\LaTeX].

When compiling your own programs that make use of the routines in the libraries 
of this project, you will in general also have to link to the necessary 
\texttt{gsl} and \texttt{root} libraries.  For examples see the makefiles that 
can be found in all the subdirectories in the \texttt{progs} directory.  


\subsection{Troubleshooting}
\begin{itemize}
\item Error in \texttt{cmake}: ROOTSYS not set.
If you get an error like this you probably should add the path of the 
\texttt{root-config} script to the search path in the file 
\texttt{\$\{PROJECT\_SOURCE\_DIR\}/cmake/Modules/FindROOT.cmake}. This is done 
by adding the path to the line \texttt{SET(ROOT\_CONFIG\_SEARCHPATH 
/insert\_your\_path\_here)}. In my case this was \texttt{usr/local/bin}. If you 
don't know the location of the \texttt{root-config} executable you maybe able to 
find it by executing \texttt{\$ which root-config}.
\item Error \texttt{Wrong ELF class}. If you get errors mentioning ELF-class then you are probably linking 64 to 32 bit code which does not compute. Make sure you are linking to the 64 bit libraries of \texttt{root} and not the 32 bit ones. For me the 64 bit libraries were located in \texttt{/usr/local/lib/root} while the 32 bit ones were in \texttt{/usr/local/root/lib/root}. 
\end{itemize}

