
\section{Integration}

The integration routines are all defined in \texttt{numint/numint.hpp}. A typical declaration of a integration routine will look like
\begin{lstlisting}
int quad_romb(const function<dcomplex> &f, double a, double b, double epsabs, double epsrel, dcomplex &result, unsigned &neval);
\end{lstlisting}
The \texttt{int} return value indicates the ``succes'' of the integration. The above example is for a one dimensional function, for multidimensional functions the boundaries \texttt{double a} and \texttt{double b} will be replaced by arrays. What about the \texttt{function<dcomplex>}? . The \texttt{struct}'s \texttt{function<typename T>} and \texttt{mdfunction<typename T, unsigned N>} for multidimensional functions are defined in \texttt{typedef.hpp},
\begin{lstlisting}
template<typename T, unsigned N>
struct mdfunction {
  /// a function with parameters
  /// @param x the evaluation point
  /// @param param pointer to parameters
  /// @param ret return value
  void (* func) (const numint::array<double,N> &x, void *param, T &ret);
  /// the parameters
  void * param;
};
\end{lstlisting}
The definition of the \texttt{struct} \texttt{function} is completely analogue except that the coordinate array \texttt{x} is replace by a single double and \texttt{template<typename T, unsigned N>} by \texttt{template<typename T>}. Note that the actual implementation of the function \texttt{func} will have to be supplied by the user as it is the integrand one wishes to integrate. Hence your integrand function will have to comply with the prototype \texttt{ myIntegrand(const numint::array<double,N> \& x, void *param, T \& ret)} for a multidimensional integrand.
\subsection{Array}
A thing that you might have noticed is usage of a custom array class \texttt{numint::array}. The full implementation of this class can be found in \texttt{array.hpp}. This class is basically an extension of the normal arrays in \texttt{c++}. It supports basic mathematical operations such as addition and multiplication. Also a \texttt{size} field is available. An important remark is that no out of bound checks are performed even though the \texttt{size} field is present.