\documentclass[11pt]{article}

\usepackage{amsmath}
\usepackage{amssymb}
\usepackage{amsfonts}
\usepackage{graphicx}
\usepackage{listings}
\usepackage{color}


\definecolor{dkgreen}{rgb}{0,0.6,0}
\definecolor{gray}{rgb}{0.5,0.5,0.5}
\definecolor{mauve}{rgb}{0.58,0,0.82}
\lstset{frame=tb,
  language=c++,
  aboveskip=3mm,
  belowskip=3mm,
  showstringspaces=false,
  columns=flexible,
  basicstyle={\small\ttfamily},
  numbers=none,
  numberstyle=\tiny\color{gray},
  keywordstyle=\color{blue},
  commentstyle=\color{dkgreen},
  stringstyle=\color{mauve},
  breaklines=true,
  breakatwhitespace=true
  tabsize=3
}

\setlength{\parindent}{0pt}

\begin{document}

\begin{center}
\rule{\textwidth}{2pt} \\ [-0.8\baselineskip]
\rule{\textwidth}{1pt} \\ \vspace{8pt}
{\Large \textsc{Introduction to my C++ code used to compute spectator tagging 
cross sections on light nuclei}} \\
\vspace{10pt}
\textsc{Wim Cosyn \texttt{wim.cosyn@ugent.be}}
\end{center}
\rule{\textwidth}{1pt} \\ [-0.75\baselineskip]
\rule{\textwidth}{2pt} \\ [2\baselineskip]

\section{Install}
You can check out the code by doing (with \texttt{<codedir>} the directory 
where 
you want the code to end up)
\begin{align*}
	\texttt{\$ svn checkout https://ssftrac.ugent.be/svn/wim/tags/LDRD-0.01 
<codedir>}
\end{align*}

\subsection{Prerequisites}
You need the following packages to successfully compile and run the code:
\begin{itemize}
 \item \texttt{cmake}\quad version at least 2.8.10
\item \texttt{ROOT}\quad version at least 5.00, packages \texttt{MathMore} and 
\texttt{Minuit2} (or \texttt{Minuit}) need to be installed
\item \texttt{gsl}
\item \texttt{doxygen, graphviz} \quad needed to build the documentation
\end{itemize}
During 
configuration, \texttt{cmake} will check all this and throw a fatal error 
when one of these is not present.

\subsection{Building the libraries}
To build 
and install the code do (in the dir where you checked 
out the code)
\begin{align*}
    & \texttt{\$ cd build} \\
	& \texttt{\$ cmake -DCMAKE\_CXX\_COMPILER=g++}\\
 &\qquad \texttt{-DCMAKE\_C\_COMPILER=gcc 
-DCMAKE\_BUILD\_TYPE=Release ..} \\
	& \texttt{\$ make -jN}\quad [\texttt{N}= \text{number of CPU cores you 
want 
to use during compilation}]\\
	& \texttt{\$ make install } \\
	& \texttt{\$ make doc } 
\end{align*}
If you have the intel C and C++ compilers installed and want to use those, 
substitute \texttt{g++} 
with \texttt{icpc} and \texttt{gcc} with \texttt{icc}.  Other build types than 
\texttt{Release} include \texttt{Debug, Profile} (there's more, see the 
\texttt{CMakeLists.txt} file in the \texttt{<codedir>} for those if you want to 
use 
them). 
If your hostname is not one of 
the predefined ones (my own pc \& laptop, the HPC of UGent) you will end up 
using the default compiler flags, which should be fine in most cases.  In case 
you want to add your custom compiler flags, you can add an extra case in the 
\texttt{CMakeLists.txt} file in the root dir of the code.  The last command 
(\texttt{\$ make doc}) generates the API through DoxyGen.  These can be 
accessed in 
the \texttt{<codedir>/doc/html/index.html} file [html] or by compiling 
\texttt{<codedir>/doc/latex/refman.text} [\LaTeX].

When compiling your own programs that make use of the routines in the libraries 
of this project, you will in general also have to link to the necessary 
\texttt{gsl} and \texttt{root} libraries.  For an example see the makefile 
that is included with the example program in 
\texttt{<codedir>/progs/LDRD\_test} (though linking to gsl is not really 
necessary there to compile the program)


\section{Troubleshooting}
\begin{itemize}
\item Error \texttt{Wrong ELF class}. If you get errors mentioning ELF-class 
then you are probably linking 64 to 32 bit code which does not compute. Make 
sure you are linking to the 64 bit libraries of \texttt{root} and not the 32 bit 
ones. For me the 64 bit libraries were located in \texttt{/usr/local/lib/root} 
while the 32 bit ones were in \texttt{/usr/local/root/lib/root}. 
\item Other troubles? Contact me through e-mail (wim.cosyn@ugent.be)
\end{itemize}

\section{Tagged spectator DIS}
The necessary subroutines for tagged spectator DIS off a deuteron or $^3$He 
target can be found in the \texttt{DeuteronCross} and \texttt{He3Cross} classes 
in the \texttt{DIS} library.  These subroutines include
\texttt{\begin{itemize}
\item DeuteronCross::getavgLCCross
\item DeuteronCross::getLCCross
\item DeuteronCross::getavgVNACross
\item DeuteronCross::getVNACross
\item He3Cross::getCross
\end{itemize}}
For the lightcone formalism, the FSI computation is currently still being 
tested so cannot be used.  More information on these functions can be found in 
the API (see above 
how to generate it and where to find it).  Each of the above functions takes an 
instance of \texttt{LightConeKin2to2} or \texttt{LightConeKine2to3} as input 
which contains the complete kinematics of a 2 to 2 (deuteron) or 2 to 3 
($^3$He) event.  I can write additional constructors for this class if these 
would be more convenient to integrate them in the event generators.  Now they 
take as input the momentum three vectors of the initial and detected particles 
and their masses.  

For the $\sigma_{\text{tot}}$ rescattering parameter in the 
FSI, to use the value we obtained from our fits to the Deeps data, call the 
\texttt{DeuteronCross::sigmaparam} and 
\texttt{DeuteronCross::setScatter} functions (see API for more information).

A sample program is included in \texttt{<codedir>/progs/LDRD\_test} that 
demonstrates the usage of all the above routines.  Simply run \texttt{\$ make} 
in that directory to compile the program.  To run it do \texttt{\$ 
<codedir>/bin/LDRD\_test <codedir>/share}.  The output should look like this

\texttt{
deuteron 809001 703071 124961 108599 574065 88672.1
\newline\newline
changing sigma parameter \newline
new sigma [mb]: 329.055 \newline
159180 24587.5 \newline
\newline
He3 31.9164 \newline
}

\end{document}
